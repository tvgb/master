\chapter{Evaluering av løsningen}
I denne delen diskuteres det rundt løsningen som ble laget, altså mobilapplikasjonen Sauron. Kapitlet går gjennom utforming av design, kodebasen og backend og det diskuteres rundt valgene som ble tatt og fordeler og ulemper med disse sammenlignet med andre løsninger.

\section{Utforming/design}
Designet av applikasjonen ble utført i en egen designfase og ble gjort i sin helhet i design- og prototypeverktøyet Figma. Dette har ført til at applikasjonen har et helhetlig og konsist design med god lesbarhet. Likevel kan vi ikke utelukke at designet kunne blitt bedre om det hadde blitt kjørt to iterasjoner rundt designprosessen med brukertester. Da kunne vi tidligere ha fått tilbakemeldinger på designet som vi nå ikke fikk før vi hadde kjørt en full brukertest av hele applikasjonen.

\section{Kodebasen}
Som det kommer fram i underkapittel \ref{sec:tilstandshandtering} \nameref{sec:tilstandshandtering}, har koden blitt strukturert på en måte som potensielt vil gjøre det enkelt å utvide applikasjonen i framtiden. Å oppnå dette har vært et mål helt siden fordypningsprosjektet ettersom det var planlagt å utvide funksjonaliteten allerede under masteroppgaven når applikasjonen skulle fullføres. Ulempen med den valgte arkitekturen er at den er abstrakt og skaper økt kompleksitet og kan derfor gjøre det vanskelig for nye utviklere å sette seg inn i.
\newline

\noindent
Ved bruk av Ionic med Capacitor har det blitt utviklet en kodebase som fungerer både på Android og iOS, med nesten ingen plattform-spesifikk kode. Dette gjør at det i framtiden vil være enklere å utvide eller bugfikse applikasjonen, ettersom at man bare må gjøre endringer én gang for å endre både iOS- og Android-versjonen. Likevel var det enkelte deler av koden som måtte utvikles spesifikt for hver av plattformene. For at tekst-til-tale skulle fungere som ønsket under blind registrering av en saueflokk ble det nødvendig å utvikle et programtillegg for Capacitor til akkurat dette. Et slikt programtillegg utvikles i Java Android for Android og Swift for iOS og bindes sammen ved hjelp av Capacitor. Dette betyr at om det blir gjort endringer i rammeverkene for tekst-til-tale for enten Android eller Swift vil man være nødt til å oppdatere dette programtillegget. Heller ikke for registrering av GPS-lokasjon i bakgrunnen fantes det et programtillegg til Capacitor som fungerte som ønsket. Forsøket på implementasjonen av bakgrunnsoppdatering av GPS-lokasjonen ble gjort ganske sent i prosjektet og det ble da besluttet å ikke utvikle et eget programtillegg for bakgrunnsoppdateringer grunnet tidsrestriksjoner. Ved bruk av et annet rammerverk enn Ionic, som for eksempel React Native som har bedre støtte for native-funksjonalitet, ville dette problemet sannsynligvis vært unngått.

\section{Backend}
Ved å bruke en løsning fra Firebase for både autentisering og lagring i database i form av Backend-as-a-Service ble det ikke nødvendig å utvikle en spesifikk tjenerløsning. Med funksjonaliteten som per nå er utviklet tilbyr Firebase de tjenestene som er nødvendige uten kompromiss. Hvis det i framtiden skulle være ønskelig med mer funksjonalitet i backend enn det Firebase muliggjør, ville man potensielt måtte flytte noe av logikken over på en dedikert tjener. Likevel kan mesteparten av logikken for autentisering og lagring kunne beholdes hos Firebase da de tilbyr løsninger for integrasjon med egne tjener-løsninger.