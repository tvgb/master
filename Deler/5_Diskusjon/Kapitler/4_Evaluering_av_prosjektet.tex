\chapter{Evaluering av prosjektet}
Dette kapitlet tar for seg prosjektet og hvorvidt resultatene som ble funnet kan brukes til å besvare forskningsspørsmålet stilt i underkapittel \ref{sec:problemstilling} \nameref{sec:problemstilling}: \textit{F1: Hvordan utvikle et digitalt verktøy for å bistå saubønder, beitelag og gjetere på oppsynstur slik at arbeidet med manuell registrering blir mer effektivt og raskere?} Dette spørsmålet ble deretter delt opp i fire underspørsmål som til sammen skal besvare hovedspørsmålet. Det er disse spørsmålene vi forsøker å besvare i teksten under.

\section{F1.1: Kan man lage et system som erstatter dagens løsning med penn og papir, men fortsatt dekker alle brukerens behov?}
Professor Hvasshovd har over prosjektets løp presentert hvordan dagens løsning for registrering av sau på beite med penn og papir foregår og hvilke behov som må tilfredstilles med en digital løsning. Et av de viktigiste behovene som Sauron dekker er at ruten for oppsynsturen blir automatisk sporet underveis og at observasjoner av sau kan krysses direkte i kartet med stor nøyaktighet. Det største utfordringen med en digital løsning som Sauron sammenlignet penn og papir er mobilnettets dekning og mobiltelefonens strømforbruk på lengre turer. Ved å laste ned kartutsnitt via internett på forhånd resulterer i at Sauron bare er avhengig av GPS for å hente inn lokasjonen til brukeren og dermed ikke er avhengig av internettdekning. Siden applikasjonen henter inn GPS-lokasjon mens applikasjonen er aktiv vil dette tære på mobilens strømforbruk. Selv om det ikke var mulig å implementere at GPS-lokasjonen ble hentet når applikasjonen var i bakgrunnen, ble det implementert strømsparingsmodus mens applikasjonen var aktiv for å minimere strømforbruket så mye som mulig. Det anses derfor som at dette målet er nådd, selv om det må tas forbehold om at det ikke har blitt testet i en reell oppsynstur på faktisk utmarksbeite. De resterende behovene til brukerne blir diskutert i de neste delmålene. 

\section{F1.2: Hvordan kan et digitalt system bistå sauebonde, beitelag og gjetere slik at de kan samhandle og dele informasjon om oppsynsturer med hverandre og norske myndigheter?}
Et av målene med prosjektet har vært å lage et helhetlig system, som i motseting til dagens løsning, gjør det enkelt å dele informasjon om oppsynsturer innad i et beitelag og med myndighetene. Systemet som er laget bistår med informasjonsdeling innad i beitelaget. Oppsynsturer som blir registrert er tilgjengelig for bønder innenfor samme beitelag. Dette har vi løst ved at oppsynsturene lagres i en database i skyen ved hjelp av Firebase. Ved hjelp av logikk på tjenersiden sørges det for at bøndene bare får tilgang til de oppsynsturene som er registrert av medlemmer innenfor samme beitelag.
\newline
\newline
\noindent
Det har ikke blitt utviklet funksjonalitet for å dele ferdiglagde oppsynsturrapporter med myndighetene. Applikasjonen vil likevel kunne hjelpe med å lage rapporter da systemet vil i motsetning til det eksisterende løsningen med penn og papir, samle all informasjon for utførte oppsynsturer fra et helt beitelag på et sted. Det vil potensielt gjøre det enklere å lage en mer fullstendig rapport, da man vil ha tilgang på mer informasjon gjennom applikasjonen enn et som var tilgjengelig tidligere. I tillegg vil det bli enklere for myndighetene å gå gjennom rapportene dersom mange beitelag benytter seg samme løsning ettersom at rapportene blir sendt inn med identisk format. 

\section{F1.3: Hvordan utvikle et brukergrensesnitt som muliggjør registrering av sau uten å måtte se på mobilskjermen?}
Under fordypningsprosjektet ble det utforsket flere ulike metoder for hvordan man kan lage et brukergrensesnitt som gjør det mulig for en gjeter/bonde å bruke en kikkert til å inspisere en saueflokk, mens man samtidig registrerer informasjonen på mobilen. Brukergrensesnittet som skulle lages måtte i praksis kunne brukes helt blindt. Grensensittet som ble utviklet bruker kombinasjon av haptisk tilbakemelding, tekst-til-tale og ulike interaksjonsmetoder som sveiping, vanlig trykking og lengre tykk for å oppnå dette. Med bakgrunn i undersøkelsene som ble utført under fordypningsprosjektet \cite{Abtahi2020TilsynBeite} vil vi påstå at grensesnittet som er laget, i teorien, vil fungere til å registrere informasjon om saueflokker mens man samtidig benytter seg av kikkert. Likevel hadde det vært ønskelig å kunne utføre en ekte test ute i feltet under beitesesongen med en person som er vant til å gå oppsynsturer, for å sjekke hvordan applikasjonen fungerer i en reell setting.

\section{F1.4: Hvordan utvikle et brukergrensesnitt som muliggjør registrering av sau under alle værforhold?}
Egen mini brukertest? 