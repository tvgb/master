\chapter{Problemstilling og mål}
I dette kapittelet skal målet og problemstillingen for masteroppgaven presenteres. Formålet med masteroppgaven blir å svare på målet og problemstillingene gjennom denne rapporten. 

\section{Brukerscenario}
For å enklere sette seg inn i dagens situasjon og utforske problemene som oppstår under oppsynsturer, har det blitt utviklet et brukerscenario som illustrerer behovet for og målet med en digital løsning.  
\newline
\newline
Geir er sauebonde og er med i et beitelag som samarbeider med å slippe sau på utmarksbeite. Beitelaget har ansvar for tilsammen 13 sauer og 17 lam. I dag er det hans tur å dra på den ukentlige oppsynsturen for å sjekke hvor sauene er og om de har det bra. Han pakker med seg kart, kikkert og litt papir. På kartet har han markert posisjonen der saueflokken  sist ble observert på forrige oppsynstur og tar utgangspunkt i denne lokasjonen for å finne sauene. Idet han kommer frem til stedet, som er en dal, oppdager han at noen av sauene vandrer rundt på en bakketopp på andre siden av dalen. Det betyr at saueflokken har delt seg opp i flere mindre grupper. Han henter frem kikkerten og forsøker først å telle totalt antall sauer han ser, men sauene beveger seg ut av syne før han rekker å telle alle og notere ned antallet på papiret med penn. Etter å ha forsøkt å telle en liten stund, ser han tilsammen 23 sauer og lam, og denne gangen klarer han å notere ned hvor mange sauer det er og hvilke farger de har på papiret samtidig som han observerer med kikkert. Fargene på sauene hjelper han med å forstå hvilken gruppe av sauer han mangler. Han må også huske på å notere ned på kartet hvor han har fått øye på sauene. Selv med kikkert er avstanden for stor til at Geir klarer å se om sauene er søye eller lam, og fargen på eventuelle bjelleslips de har på seg. Han går litt videre før han får øye på et par sauer, men nok en gang forsvinner dem i bakketoppene idet Geir legger vekk kikkerten for å skrive på papiret sitt. Nå begynner Geir å bli skikkelig frustert over at sauene beveger seg mens han skal notere. Etter å ha ventet på at sauene skal komme til syne nok en gang, klarer endelig Geir å få øye på resten at flokken og han noterer kjapt ned antall og farge på sauene før de forsvinner igjen. Med riktig antall sauer notert, går han tilbake til gården sin og prøver å skrive ned ruten han gikk for å finne sauene. Geir legger til notatene for dagens oppsynstur i et dokument der beitelaget har samlet dato og informasjon for sine gjennomførte oppsynsturer denne sommeren. Når beitesesongen er over, vil beitelaget lage en samlet rapport med informasjonen alle gjeterne har samlet fra alle oppsynsturene og sende dem til Statsforvalteren. 

\section{Mål for oppgaven} \label{sec:mal-for-oppgaven}
Dette prosjektet er delt i en prosjektoppgave \cite{Abtahi2020TilsynBeite} som gikk over høsten 2020 og deretter denne masteroppgaven som foregikk over våren 2021 som bygger videre på prosjektoppgavens arbeid. Prosjektoppgavens mål var mer tilspisset og gikk ut på å \enquote{Designe og utvikle en applikasjon som muliggjør effektiv registrering av sauer på utmarksbeite i situasjoner der brukeren kontinuerlig benytter seg av kikkert og dermed ikke kan se på skjermen} \cite[~s.3]{Abtahi2020TilsynBeite}. Dette målet ble svart på ved å utvikle tre ulike brukergrensesnitt for registrering av sau uten å se på mobilskjermen og så sammenligne disse mot hverandre med data fra brukertester på ti personer. Under masteroppgaven skal det utvikles en fullstendig applikasjon som bygger videre på resultatene fra fordypningsprosjektet. Denne har fått navnet \textit{Sauron}. Masteroppgavens mål som er overordnede målet for hele prosjektet, som formulert som følger:
\newline
\newline
\textit{Designe, utvikle og implementere en helhetlig applikasjon som kan bistå sauebønder og beitelag med å registrere småfe på oppsynsturer.}

\section{Problemstilling} \label{sec:problemstilling}
For å kunne evaluere målet for masteroppgaven er det blitt formulert forskningsspørsmål med tilhørende underspørsmål som skal besvares i løpet av gjennomføringen av masteroppgaven: 
\begin{itemize}
    \item \textbf{F1: Hvordan utvikle et digitalt verktøy for å bistå sauebønder, beitelag og gjetere på oppsynstur slik at arbeidet med manuell registrering blir mer effektivt og raskere?}
    \item \textbf{F1.1: Kan man lage et system som erstatter dagens løsning med penn og papir, men fortsatt dekker alle brukerens behov?}
    \item \textbf{F1.2: Hvordan kan et digitalt system bistå sauebonde, beitelag og gjetere slik at de kan samhandle og dele informasjon om oppsynsturer med hverandre og norske myndigheter?}
    \item \textbf{F1.3: Hvordan utvikle et brukergrensesnitt som muliggjør registrering av sau uten å måtte se på mobilskjermen?}
    \item \textbf{F1.4: Hvordan utvikle et brukergrensesnitt som muliggjør registrering av sau under alle værforhold?}
\end{itemize}
