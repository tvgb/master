\chapter{Resultater fra brukertestene} \label{resultater}
Dette kapittelet skal presentere resultatene fra de fem brukervennlighetstestene som ble gjennomført. Det vil innebære kvalitative resultater som observasjoner av testdeltakerene mens de utfører oppgavene presisert i \ref{brukertest} og deres tilbakemeldinger til applikasjonen. 

\section{Kvalitative resultater}
Resultatene vil bli beskrevet som sammendrag av alle testdeltakernes prestasjoner og tilbakemeldinger oppgave for oppgave. Avslutningsvis vil alle resultatene vises i en tabell som fremhever oppgavene der testdeltakerne støtte på problemer i brukertestene. 

\subsection{Oppgave 1: Laste ned kartutsnitt}
De aller fleste testdeltakerne klarte å utføre oppgave 1 uten problemer, både det å laste ned kartutsnittet og endre navnet på utsnittet etterpå. For en deltaker tok det svært lang tid for applikasjonen å vise kartutsnittet som var tilgjengelig for nedlasting, noe som førte til at deltakeren trykket på \enquote{Last ned}-knappen før kartet ble vist på skjermen. Dermed måtte deltakeren ut av siden for nedlasting av kart og så tilbake igjen for å få kartet til å laste. Samme deltaker gikk også ut av siden og klikket rundt på andre knapper da personen fikk i oppgave om å endre navnet på kartsiden. Etter at forsøksleder presiserte at det var kartutsnitt og ikke oppsynstur som skulle endres navn på, klarte deltakeren å gjennomføre oppgaven.

\subsection{Oppgave 2: Ny oppsynstur}
Alle testdeltakerne klarte å opprette en ny oppsynstur og la til alle medlemmene som var med på oppsynsturen uten å støte på problemer. En av brukerne la ikke merke til at personen allerede var lagt inn på oppsynsturen og la dermed til seg selv på nytt i tillegg til forsøksleder og observatør. Fant selv ut at personen var lagt til dobbelt og fjerner seg selv. Denne personen kommenterte at \enquote{appen var veldig intuitiv og jeg skjønte alt selv jeg ikke vet hva en oppsynstur ser}. 

\subsection{Oppgave 3: Registrere sauer}
Alle deltakere bortsett fra én forstod ikke at markøren på kartet er der lokasjonen til en registrering vil bli plassert, og flyttet derfor ikke kartet slik at markøren ville vise eksakt lokasjon til sauene. Tre av deltakere forstod det etter å registert sauene og så at det ble opprettet en pin der markøren var plassert, og brukte markøren på riktig måte etter hvert. En av deltakerne forstod det ikke gjennom hele brukertesten. 
\newline
\newline
En deltaker ga uttrykk for at ikonene for registrering var for små og at det var lett for å trykke på feil ikon. Deltakeren mente også at det gjerne kunne være mer luft mellom selve ikonet og sirkelen rundt for estetikkens skyld. Samme bruker gav uttrykk for at rovdyrikonet kunne forveksles med et en hundepote og burde gjøres \enquote{farligere} for å tydeliggjøre at det er ikonet for rovdyr.
\newline
\newline
For en av deltakerne ble det ikke registrert at personen trykket på symbolet for sau for å registrere og måtte trykke en gang til. Samme deltaker gikk også videre fra registrering av øremerke før det nye øremerket som personen hadde opprettet ble lagret ettersom tastaturet presset opp \enquote{Neste}-knappen opp til \enquote{Lagre}-knappen, og dermed ble det kort avstand mellom knappene og deltakeren trykket på \enquote{Neste} i stedet for \enquote{Lagre}. Deltakeren oppdaget feilen i oppsummerings-siden, gikk tilbake og rettet opp i feilen selv. En annen deltaker gikk også videre til oppsummeringssiden uten å trykke på det nylig opprettede øremerket, men det var på grunn av at deltakeren ikke fikk med seg at man måtte huke av avkrysningsboksen for å registrere øremerket og ikke grunnet tastaturet som beskrevet for forrige deltaker.  
\newline
\newline
I denne oppgaven ønsket utviklerne i undersøke om det var en forskjell på deltakerne som allerede var kjent med grensesnittet for registrering mot deltakerne som ikke hadde tidligere kunnskap. En av de nye deltakerne brukte tid på å forstå at det måtte sveipes til høyre eller venstre og ikke trykkes for å bytte underkategori av registreringen. Dette fant deltakeren ut på egen hånd etter litt prøving og feiling og deretter gikk registreringen feilfritt. Den andre deltakeren som heller ikke hadde erfaring med registreringsgrensesnittet klarte heller ikke å forstå hvordan sveipingen for å bytte kategori, og måtte til slutt få hjelp av forsøksleder for å fullføre registreringen. Det var ingen problemer med registreringen for personene som var kjente med registreringsgrensesnittet fra før. En kjent deltaker lurte på hvorfor det var piler på sidene dersom de ikke kunne trykkes på, bare sveipes. 

\subsection{Oppgave 4: Registrere rovdyr}
En av deltakerne opplevde at GPS-lokasjonen hoppet et stykke bort idet personene skulle registrere rovdyr, som førte til forvirring. Klarte likevel å fullføre registreringen slik det var ment.

\subsection{Oppgave 5: Registrere død sau}
 En av deltakerne trodde symbolet med plaster på som er ment for skadet sau var symbolet før død sau. Oppdager feilen da personen leser tittelen på siden for skadet sau og går tilbake til kartsiden. Forstår at det er dødningshodet som er symbolet for død sau og derfra går resten av registrering bra.
 \newline
 \newline
To av deltakerne prøvde å trykke på bildeikonet i overskriften for bildetaking for å ta bilde og ikke +-knappen. Begge deltakerne forstod det selv etter noe tid. 

\subsection{Oppgave 6: Registrere skadet sau}
Alle deltakerne av brukertesten utførte registrering av skadet sau feilfritt. En deltaker sa også under registreringen: \enquote{Dette er litt gøy}. 

\subsection{Oppgave 7: Fullfør oppsynstur}
Den første testdeltakeren klarte å finne oppsummeringssiden til oppsynsturen som ment, men da personen trykket på \enquote{Fullfør}-knappen kom det en tilbakemelding fra applikasjonen at det forekom en feil i opplastingen av oppsynsturen til Firebase databasen. Utviklerne måtte derfor lage en falsk oppsynstur for denne deltakeren slik at brukertesten kunne fullføres og gjennomføre siste oppgave. Etter omfattende testing ble det avdekket at feilen stammet fra hvordan bildene som ble tatt i siden for registrering av døde sauer blir lagt til i databasen. Bildene førte til at dataene fra oppsynsturen ble for store til å lastes opp i databasen (mer om dette i \ref{kritiske_feil}). For de neste brukertestene ble det besluttet å endre oppgave 5 slik at deltakerne slettet bildet som ble tatt før oppsynsturen ble fullført. Etter denne korrigeringen av oppgave 5 hadde ingen andre testdeltakere problemer med å utføre oppgave 7.
 

\subsection{Oppgave 8: Sjekk liste av oppsynsturer}
Alle deltakerne fant fram til oversikten over oppsynsturene. Derimot var det to deltakere som ikke kom inn på sin nylig registrerte oppsyntur ettersom de trykket på hele boksen for oppsynsturen og ikke pilen til høyre. Ingen av deltakerne fikk heller ikke med seg at det var mulighet for å trykke på firkanten inne i boksen for hver registrering slik at kartet automatisk flyttet seg til aktuelle pinnen. To av deltakerne kommenterte også på at det var lite mellomrom mellom tittelen på siden og kartet. Ellers var det en deltaker som kommenterte \enquote{For en fin oversikt}. 

\subsection{Generelle tilbakemeldinger}
Etter oppgavene var fullførte, spurte forsøkslederen testdeltakerne om hva slags inntrykk og generelle tilbakemeldinger de hadde om applikasjonen. Flere av testdeltakerne ga uttryk for at applikasjonen var enkel og lett å forstå, spesielt ikonene for de ulike registreringene var mulige å forstå uten tekstlige beskrivelser. Ikonene kunne derimot være større for å være enklere å trykke på. En av deltakerne synes det var vanskelig å se alternativene på enkelte av dialogboksene grunnet lav kontrast på bakgrunnsfargen og tekstfargen. Tekstskriften på knappene ble også kommentert på for å være for tynn og dermed vanskelig å lese. En annen deltaker delte at systemet burde lagre informasjonen i oppsynsturene regelmessig til fil for å ta vare på data dersom mobilen skulle gå tom for strøm. En av deltakerne som ikke var kjent med applikasjonen fra tidligere ønsket at det skulle være en pil eller forklaring som beskrev hvordan grensesnittet for registrering fungerte første gangen det skulle benyttes. Applikasjonens mørke tema ble roset av en testdeltaker, som også likte at all funksjonalitet er tilgjengelig fra hjemskjermen. Likte at funksjonaliteten var adskilt fra hverandre? 

\subsection{Resultatstabell} \label{resultattabell}

\begin{longtable}{| p{0.1\linewidth} | p{0.1\linewidth} | p{0.2\linewidth} | p{0.25\linewidth} | p{0.12\linewidth} | p{0.09\linewidth} |}
\caption{Tabell som viser resultater fra de utførste brukertestene.}
\label{tbl:resultater-brukertester} \\

\hline
\multicolumn{1}{|C{0.1\linewidth}|}{Deltaker nr.} &
\multicolumn{1}{C{0.1\linewidth}|}{Tidligere deltaker} &
\multicolumn{1}{l|}{Oppgave} & 
\multicolumn{1}{l|}{Problem} &
\multicolumn{1}{l|}{Tag 1} &
\multicolumn{1}{l|}{Følelse} \\ 
\hline 
\endfirsthead

\multicolumn{6}{c}
{{\bfseries \tablename\ \thetable{} -- fortsettelse fra forrige side}} \\
\hline 
\multicolumn{1}{|C{0.1\linewidth}|}{Deltake nr.} &
\multicolumn{1}{C{0.1\linewidth}|}{Tidligere deltaker} &
\multicolumn{1}{l|}{Oppgave} & 
\multicolumn{1}{l|}{Problem} &
\multicolumn{1}{l|}{Tag 1} &
\multicolumn{1}{l|}{Følelse} \\ 
\hline 
\endhead

\hline \multicolumn{6}{|r|}{{Fortsetter på neste side}} \\ \hline
\endfoot

\endlastfoot

1 & Nei & 3: Registrere sauer & Litt vanskeligheter i starten med å forstå at man skal sveipe sidelengs for å bytte underkategorier. & Registrering & Forvirret \\
\hline
1 & & 5: Registrere død sau & GPS hopper et stykke bort. & GPS & Forvirret \\
\hline
\hline
2 & Ja & 1: Laste ned kartutsnitt. & Kartet lastes inn tregt, deltaker trykker på "Last ned kartutsnitt" før kartet er ferdig innlastet. & GPS & Forvirret, irritert \\
\hline
2 & & 3: Registrere sauer. & Får litt hjelp til å zoome, flytter ikke markør på sted for eksakt lokasjon av registrering. & Markør & Forvirret \\
\hline
2 & & 3.6: Registrere øremerker. & Går videre fra siden for registrering av øremerker til oppsummeringssiden før øremerket har blitt lagret. Trykker på "Neste"-knappen framfor å lukke tastaturet og trykke lagre. & Registrering & Forvirret \\
\hline
2 & & 5.3: Registrere død sau. & Trykker på bildeikonet for å ta bilde og ikke \enquote{+}-knappen. & Bildetaking & \\
\hline
\hline
3 & Ja & 3: Registrere sauer. & Flytter ikke markøren for eksakt lokasjon for registrering. & Markør & \\
\hline
3 & & 5: Registrere død sau. & Klikker på plaster-ikonet, som er ikonet for skadet sau, først. Skjønner at det er feil og bytter selv til rett registrerings-side. & Ikon & Forvirret \\
\hline
3 & & 8.2: Sjekke liste med oppsynsturer. & Forsøker å klikke på hele boksen for å komme inn på en oppsynstur framfor å bruke pilen. & Oppsynsturer & Forvirret, irritert \\
\hline
\hline
4 & Ja & 3: Registrere sauer. & Flytter ikke markøren til eksakt lokasjon for registrering. & Markør & \\
\hline
4 & & 8.2: Sjekke liste med oppsynsturer. & Forsøker å klikke på hele boksen for å komme inn på en oppsynstur framfor å bruke pilen. & Oppsynsturer & Forvirret \\
\hline
\hline
5 & Nei & 2: Ny oppsyntur. & Legger ikke merke til at personen allerede er lagt inn som deltaker av oppsynsturen og legger til seg selv på nytt. Oppdager selv at det er registrert dobbelt og fjerner seg selv fra listen slik at det bare står en gang. & Ny oppsynstur & \\
\hline
5 & & 3: Registrere sauer. & Flytter ikke markøren for eksakt lokasjon for registreringen. & Markør & \\
\hline
5 & & 3.6: Registrere øremerker. &  Sjekker ikke av boksen for å velge øremerket som personen akkurat har laget. Ser det heller ikke i oppsummeringssiden. & Registrering & \\
\hline
5 & & 5.3 Registrere død sau. & Trykker på bildeikonet for å ta bilde og ikke \enquote{+}-knappen. & Bildetaking & Forvirret \\
\hline
\end{longtable}

\section{Konklusjon}
 (Det ble utført 5 kvalitative brukertester som beskrevet i kapt. \ref{brukertest}. Ved første brukertest ble det oppdaget at bildene som ble tatt under brukertesten førte til at oppsynsturen ikke kunne bli lastet opp på skyen, som førte til at brukertesten ble endret slik at bildene ikke ble lagt til for de neste brukertestene. Brukertestene avslørte at flere deltakere ikke forstod at markøren for registrering kunne flyttet på til eksakt lokasjon for observasjonen før et stykke ut i brukertesten. DETTE MÅ DU KANSKJE SKRIVE OM ETTER ANDRE Brukerne som ikke var kjent med applikasjonen fra før brukte mer tid på å forstå hvordan brukergrensesnittet med sveiping fungerte enn dem som var kjent med det fra før. Ellers var det gode tilbakemeldinger om at applikasjonen var enkel å forstå og i bruk. )




