\chapter*{Sammendrag}
Rundt to millioner sauer slippes på utmarksbeite i sommerhalvåret hvert år. Dette er ikke uten risiko, da omlag 3-7\% av sauene mister livet som følge av sykdom, ulykker eller rovdyr i løpet av denne perioden. For å sikre velferden til sauene kreves det fra norske myndigheter at sauebonden gjennomfører en oppsynstur minst én gang i uken. Formålet med oppsynsturen er å kartlegge hvor sauene er på beitet, hvordan de beveger seg og å oppdage eventuelle skadde eller døde sauer. Per dags dato utføres slike oppsynsturer ved at sauebonden drar ut på beitet og noterer ned den nødvendige informasjonen om sau og lokasjon ved hjelp av penn og papir. Det var derfor et ønske om å få utviklet en mobilapplikasjon som kan bistå med denne oppgaven. Mobilapplikasjonen burde kunne brukes blindt og med bare en hånd ettersom at det ofte tas i bruk kikkert mens man er på oppsynstur. Det var også et ønske om å få på plass funksjonalitet som vil gjøre det enklere å dele informasjonen man har samlet inn med andre bønder innenfor eget beitelag. 
\newline
\newline
Prosjektet i sin helhet har bestått av to deler, et fordypningsprosjekt og selve masteroppgaven. I fordypningsprosjektet ble det gjort et dypdykk i eksisterende teknologi og metoder for utvikling av brukergrensesnitt som kan brukes blindt. Med bakgrunn i funnene fra litteratursøket ble deretter tre forskjellige grensenitt utviklet og brukertestet for å finne ut hva som fungerer best for å registrere sau under en oppsynstur. Resultatene fra fordypningsprosjektet ble brukt for å lage et grensesnitt som til slutt ble implementert i den endelige versjonen av applikasjonen.
\newline
\newline
Masteroppgaven startet med et litteratursøk for å kartlegge bondens behov og dagens utfordringer i forbindelse med informasjonsregistrering under en oppsynstur. Det ble deretter laget en kravspesifikasjon med bakgrunn i dette og i dialog med Professor Hvasshovd. En første versjon av applikasjonen ble så designet og utviklet. Den utviklede applikasjonen er kryssplattform for mobile enheter som kjører Android eller iOS. Applikasjonen har ingen dedikert tjenerløsning, men bruker Firebase som Backend-as-a-Service. I slutten av prosjektet ble det gjennomført brukertester på fem ulike brukere. Resultatene fra testene ble deretter analysert og brukt til å komme med forslag til potensielle endringer som kan gjøres for å gjøre applikasjonen mer brukbar.
\newline
\newline
Resultatene fra brukertestene viser at det er mulig å lage en applikasjon som kan brukes på oppsynsturer hvor man må registrere informasjon om sauflokker uten å ha mulighet til å se på skjermen. Informasjonen som kan registreres gjennom applikasjonen er mer enn tilstrekkelig med hensyn til framstillingen av rapporten som kreves av norske myndigheter. Den implementerte innloggings- og databaseløsningen gjør det også mulig for bønder og gjetere innenfor samme beitelag å dele oppsynsturer med hverandre. Sammenlignet med dagens løsning kan applikasjonen gi bønder og gjetere en enklere arbeidshverdag og bidra til at færre sau dør i løpet av året grunnet mer nøyaktig registrering av informasjon og bedre informasjonsflyt innad i beitelag.

\chapter*{Abstract}
Approximately two million sheep are released for free range pasture during the summer in Norway every year. This is not without risk, as about 3-7\% of the sheep lose their lives because of illness, accidents or predators during this period. To ensure the welfare of the sheep, Norwegian authorities require that sheep farmers carry out inspections of the herd at least once a week. The purpose of these inspections is to be able to map where the sheep are in the pasture, how they move and to find potential dead or injured sheep. To this day, farmers carry out these inspections in the pasture, writing down necessary information about sheep and location using pen and paper. It was therefore a desire to develop a mobile application that would be able to assist with this task. The mobile application should be able to be used blindly and one handed as binoculars are often used during the pasture inspection. There was also a desire to implement functionality that would make it easier to share the gathered information with other farmers.
\newline
\newline
The project has in its entirety consisted of two parts, a research project and this master’s thesis. The research project started with a deep dive into existing technology and methods used for developing user interfaces that can be utilized without having to look at the screen. With the knowledge gained from the literature review, three different user interfaces were created and tested on users to find out what worked the best for this specific use case. The results from the research project were then used to create a user interface that was implemented into the final version of the application.
\newline
\newline
The master’s thesis started with a literature review to identify the farmers needs regarding information registration during a pasture inspection. From this, user requirements for the application were created in collaboration with Professor Hvasshovd. The first version of the application was then designed and developed. The developed application is a cross platform application for mobile devices running either Android or iOS. The application has no dedicated backend solution but uses Firebase as Backend-as-a-Service. In the final stages of the project, user tests were conducted on five different users. The results for the users tests were then analysed and used to propose suggestions to potential changes that can be made to make the application more usable.
\newline
\newline
The results from the user tests show that it is possible to create an application that can be used during pasture inspections where one might have to register information without being able to look at the screen. The information that can be registered through the application is more than sufficient with regard to creating the report required by the Norwegian authorities. The implemented login and database solution makes it possible for farmers and shepherds to cooperate and share information about pasture inspections with each other. Compared to today’s solution, the application can give farmers and shepherds an easier day at work and contribute to less sheep casualties each year due to more accurate registered information and increased information flow between farmers.
